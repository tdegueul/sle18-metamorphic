% Usual suspects
\newcommand*{\ie}{i.e.,\@\xspace}
\newcommand*{\eg}{e.g.,\@\xspace}
\newcommand*{\cf}{cf.\@\xspace}

\makeatletter
\newcommand*{\etc}{%
	\@ifnextchar{.}%
	{etc}%
	{etc.\@\xspace}%
}
\makeatother

\newcommand*{\de}{$\Delta$\@\xspace}
\newcommand*{\ds}{$\Delta$s\@\xspace}
\newcommand*{\db}{$\Delta$-based\@\xspace}
\newcommand*{\M}{\mathcal{M}}

\newcommand*{\diff}{\textit{diff}\@\xspace}
\newcommand*{\patch}{\textit{patch}\@\xspace}
\newcommand*{\prism}{\textsc{Prism}\@\xspace}

% Listings
\definecolor{keywordscolor}{RGB}{127, 0, 85}
\definecolor{stringcolors}{RGB}{42, 0, 255}
\definecolor{commentscolor}{RGB}{63, 127, 95}
\definecolor{annotationscolor}{RGB}{100, 100, 100}
\definecolor{lstbgcolor}{RGB}{245, 245, 245}

% Custom Java
\lstset{
	language=Java,
	%	mathescape=true,
	literate={->}{$\rightarrow$}{1},
	keywordstyle=\color{keywordscolor}\bfseries,
	commentstyle=\color{commentscolor},
	stringstyle=\color{stringcolors},
	basicstyle=\ttfamily\tiny,
	captionpos=b,
	numbers=left,
	%	backgroundcolor=\color{lstbgcolor},
	%	framexleftmargin=20pt,
	xleftmargin=17pt,
	aboveskip=\bigskipamount,
	belowskip=\bigskipamount,
	%	frame=tb,
	tabsize=2,
	breaklines=true
}

% Rascal
\lstdefinelanguage[]{Rascal}[]{Java}
{
	morecomment=[l]{\@},
	morekeywords={alias, tuple, lrel, data, str, value, int, list}
}

\usetikzlibrary{shapes.callouts, shadows}
\tikzset{author comment/.style={draw, fill=white, thick, drop shadow}}

\newcommand{\Comment}[3]{%
	\ifthenelse{\boolean{CommentON}}{%
		\raisebox{-.5ex}
		{\tikz
			\node[x=1ex, y=1ex, inner sep=.5ex,
			rectangle callout,
			callout pointer width=.7ex,
			callout relative pointer={(1.5,-0)},
			author comment]
			{\footnotesize\textsf{#1}};}~%
		\textsf{[}\,\textcolor{#2}{#3}\,\textsf{]}\xspace
	}{} %else
}

\newcommand{\td}[1]{\Comment{TD}{blue}{#1}}
\newcommand{\fc}[1]{\Comment{FC}{red}{#1}}
